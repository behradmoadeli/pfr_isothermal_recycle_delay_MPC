\section{Observer Design}
One important issue of DPSs is the limited access to the states of the infinite-dimensional system as the state is distributed over the entire domain and performing infinite measurements is never feasible. Therefore, an observer is required to estimate the states of the system based on the available measurements. To address this issue, a Luenberger observer is designed to reconstruct the states of the system based on the output measurements. First, the continuous-time observer design is considered; followed by the design of the discrete-time observer.

\subsection{Continuous-Time Observer Design}
For the purpose of state reconstruction of a diffusion-convection-reaction system, where the feedforward term $\mathfrak{D}$ is generally absent, the continuous-time observer dynamics are given by \eqref{eq:observer_continuous}.
\begin{equation} \label{eq:observer_continuous}
    \begin{aligned}
        \dot{\underline{\hat{x}}}(\zeta, t) &= \mathfrak{A} \underline{\hat{x}}(\zeta, t) + \mathfrak{B} u(t) + \mathfrak{L}_c [y(t) - \hat{y}(t)] \\
        \hat{y}(t) &= \mathfrak{C} \underline{\hat{x}}(\zeta, t)
    \end{aligned}
\end{equation}
where $\underline{\hat{x}}(\zeta, t)$ is the reconstructed state of the original system and $\mathfrak{L}_c$ is the continuous-time observer gain. By subtracting the observer dynamics from the original system dynamics, the error dynamics $e(\zeta,t$) are obtained as shown in \eqref{eq:observer_error_continuous}.
\begin{equation} \label{eq:observer_error_continuous}
    \begin{aligned}
        \dot{e}(\zeta, t) &= (\mathfrak{A} - \mathfrak{L}_c \mathfrak{C}) e(\zeta, t) \equiv \mathfrak{A}_o e(\zeta,t) \\
    \end{aligned}
\end{equation}
The goal is to design the observer gain $\mathfrak{L}_c$ such that the error dynamics are exponentially stable, i.e. $\max\{\operatorname{Re}(\lambda_{o})\}~<~0$ where $\{\lambda_{o}\}$ is the set of eigenvalues of the error dynamics operator $\mathfrak{A}_o$. Three different forms of the observer gain are considered as spatial functions $\mathfrak{L}_c = f(\zeta, l_{obs})$ with the effect of the scalar coefficient $l_{obs}$ on $\max\{\operatorname{Re}(\lambda_{o})\}$ shown in Fig.~\ref{fig:L_vs_lambda}.

\begin{figure}[!htbp]
    \centering
    \includesvg[inkscapelatex=false, height=0.23\textwidth, keepaspectratio]{figures/obs_lambda.svg}
    \caption{The effect of various observer gains $\mathfrak{L}_c = f(\zeta, l_{obs})$ on the eigenvalues of state reconstruction error dynamics $\lambda_o$.}
    \label{fig:L_vs_lambda}
\end{figure}

\subsection{Discrete-Time Observer Design}
Once an appropriate continuous-time observer gain is determined, the discrete-time observer gain $\mathfrak{L}_d$ may be obtained using the same Cayley-Tustin time discretization approach, as shown in \eqref{eq:observer_discrete}.
\begin{equation} \label{eq:observer_discrete}
    \begin{aligned}
        \underline{\hat{x}}(\zeta, k) &= \mathfrak{A}_d \underline{\hat{x}}(\zeta, k-1) + \mathfrak{B}_d u(k) + \mathfrak{L}_d [y(k) - \hat{y}(k)] \\
        \hat{y}(k) &= \mathfrak{C}_{d,o} \underline{\hat{x}}(\zeta, k-1) + \mathfrak{D}_{d,o} u(k) + \mathfrak{M}_{d,o} y(k)
    \end{aligned}
\end{equation}
with $\mathfrak{A}_d$ and $\mathfrak{B}_d$ defined in \eqref{eq:discrete_AB}, and $\mathfrak{C}_{d,o}$, $\mathfrak{D}_{d,o}$, $\mathfrak{M}_{d,o}$, and $\mathfrak{L}_d$ are given in \eqref{eq:observer_discrete_CDLM}.
\begin{equation} \label{eq:observer_discrete_CDLM}
    \begin{aligned}
        \mathfrak{C}_{d,o} (\cdot) &= \sqrt{2\alpha} \left[ I + \mathfrak{C} (\alpha I - \mathfrak{A}) \mathfrak{L}_c \right]^{-1} \mathfrak{C} \mathfrak{R}(\alpha, \mathfrak{A}) (\cdot) \\
        \mathfrak{D}_{d,o} &= \left[ I + \mathfrak{C} (\alpha I - \mathfrak{A}) \mathfrak{L}_c \right]^{-1} \mathfrak{C} \mathfrak{R}(\alpha, \mathfrak{A}) \mathfrak{B} \\
        \mathfrak{M}_{d,o} &= \left[ I + \mathfrak{C} (\alpha I - \mathfrak{A}) \mathfrak{L}_c \right]^{-1} \mathfrak{C} \mathfrak{R}(\alpha, \mathfrak{A}) \mathfrak{L}_c \\
        \mathfrak{L}_d &= \sqrt{2\alpha} \mathfrak{R}(\alpha, \mathfrak{A}) \mathfrak{L}_c \\
    \end{aligned}
\end{equation}
It can been shown that using this approach, the discrete-time error dynamics will be stable if the continuous-time observer gain $\mathfrak{L}_c$ is chosen such that $\mathfrak{A}_o$ is stable. It is also worth noting that the proposed methodology skips the need for model reduction associated with the discrete-time Luenberger observer, with no spatial approximation required as well \cite{dochain2000state,dochain2001state,alonso2004optimal,ali2015review,khatibi2021model}.