\section{Model Predictive Controller Design}

The proposed MPC, as shown in Fig.~\ref{fig:block_diagram}, is developed in this section with the goal of stabilizing the given unstable infinite-dimensional system within an optimal framework while satisfying input constraints. An infinite-time open-loop objective function sets the foundation of the controller design in the discrete-time setting at each sampling instant $k$, which consists of a weighted sum of state deviations and actuation costs for all future time instances, subject to the system dynamics and input constraints, as shown in \eqref{eq:MPC_inf_time}.

\begin{figure}[!htbp]
    \centering
    \begin{tikzpicture}[node distance=2cm, scale=0.75, transform shape]
        \node (plant) [block] {Plant};
        \node (MPC) [block, below of=plant] {MPC};
        \draw [arrow] (plant.south) -- node[midway, right] {$\underline{x}(\zeta,k)$} (MPC.north);
        \draw [arrow] (MPC.west) -- ++(-1,0) |- node[near end, above] {$u(k)$} (plant.west);
        \draw [arrow] (plant.east) -- node[midway, above] {$y(k)$} ++(1,0);
    \end{tikzpicture}
    \caption{Proposed full-state feedback model predictive control system.}
    \label{fig:block_diagram}
\end{figure}

\begin{equation} \label{eq:MPC_inf_time}
    \begin{aligned}
        \min_{U} \quad \sum_{l=0}^{\infty} &\langle \underline{x}(\zeta, k+l | k), \mathfrak{Q} \underline{x}(\zeta, k+l | k) \rangle \\
        + &\langle u(k+l+1 | k), \mathfrak{F} u(k+l+1|k) \rangle \\
        \, \\
        \text{s.t.} \quad &\underline{x}(\zeta, k+l | k) = \mathfrak{A}_d \underline{x}(\zeta, k+l-1 | k) + \mathfrak{B}_d u(k+l | k) \\
        &u^{min} \leq u(k+l | k) \leq u^{max}
    \end{aligned}
\end{equation}

where $\mathfrak{Q}$ and $\mathfrak{F}$ are positive definite operators of appropriate dimensions, responsible for penalizing state deviations and actuation costs, respectively. The notation $(k+l|k)$ indicates the future time states or input instance $k+l$ obtained at time $k$. The infinite-time optimization problem may be reduced to a finite-time setup by assigning zero-input beyond a certain control horizon $N$, resulting in the optimization problem in \eqref{eq:MPC_finite_time}.

\begin{equation} \label{eq:MPC_finite_time}
    \begin{aligned}
        \min_{U} \quad \sum_{l=0}^{N-1} &\langle \underline{x}(\zeta, k+l | k), \mathfrak{Q} \underline{x}(\zeta, k+l | k) \rangle \\
        + &\langle u(k+l+1 | k), \mathfrak{F} u(k+l+1|k) \rangle \\
        + &\langle \underline{x}(\zeta, k+N | k), \mathfrak{P} \underline{x}(\zeta, k+N | k) \rangle \\
        \, \\
        \text{s.t.} \quad &\underline{x}(\zeta, k+l | k) = \mathfrak{A}_d \underline{x}(\zeta, k+l-1 | k) + \mathfrak{B}_d u(k+l | k) \\
        &u^{min} \leq u(k+l | k) \leq u^{max} \\
        & \langle \underline{x}(\zeta, k+N | k), \underline{\phi_u}(\zeta) \rangle = 0
    \end{aligned}
\end{equation}

Obtained as the solution to the discrete-time Lyapunov equation, $\mathfrak{P}$ is the terminal cost operator as shown in \eqref{eq:terminal_cost}; which can be proven to be positive definite only if the terminal state $\underline{x}(\zeta, k+N | k)$ is in a stable subspace. Therefore, an equality constraint is introduced to guarantee that the resulting quadratic optimization problem is convex. The terminal constraint is enforced by setting the projection of the terminal state onto the unstable subspace of the system to zero \cite{curtainbook, xu2017linear, khatibi2021model}. Here, $\underline{\phi_u}(\zeta)$ is the set of unstable eigenfunctions of the system, for all eigenvalues where $\operatorname{Re}(\lambda_u) \geq 0$.

\begin{equation} \label{eq:terminal_cost}
    \mathfrak{P} (\cdot) = \sum_{m=0}^{\infty} \sum_{n=0}^{\infty} 
    -\frac{
        \langle \underline{\phi_m} , \mathfrak{Q} \underline{\psi_n} \rangle
    }{
        \lambda_m + \overline{\lambda_n}
    }
    \langle (\cdot) , \underline{\psi_n} \rangle \phi_m
\end{equation}

One may further process the optimization problem in \eqref{eq:MPC_finite_time} to obtain a standard format for quadratic programming (QP) solvers by substituting the future states in terms of the current state and the sequence of future inputs using system dynamics expression. The resulting QP problem is given in \eqref{eq:MPC_QP}. The optimal input sequence $U$ is then obtained by solving the QP problem at each sampling instant $k$. To implement a receding horizon control strategy, only the first input of the optimal sequence $u(k+1|k)$ is applied to the system, and the optimization problem is solved again at the next sampling instant $k+1$.

\begin{equation} \label{eq:MPC_QP}
    \begin{aligned}
        \min_{U} &J = U^T \langle I,H \rangle U + 2U^T \langle I, P \underline{x}(\zeta, k|k) \rangle \\
        \text{s.t.} &\qquad U^{min} \leq U \leq U^{max} \\
        &\qquad T_u \underline{x}(\zeta, k|k) + S_u U = 0
        \, \\
        \text{with } &H = \\
        &\hspace{-3.5em }\begin{bmatrix}
            \mathfrak{B}_d^* \mathfrak{P} \mathfrak{B}_d + \mathfrak{F} & \mathfrak{B}_d^* \mathfrak{A}_d^* \mathfrak{P} \mathfrak{B}_d & \cdots &  \mathfrak{B}_d^* {\mathfrak{A}_d^*}^{N-1} \mathfrak{P} \mathfrak{B}_d \\
            \mathfrak{B}_d^* \mathfrak{P} \mathfrak{A}_d \mathfrak{B}_d & \mathfrak{B}_d^* \mathfrak{P} \mathfrak{B}_d + \mathfrak{F} & \cdots & \mathfrak{B}_d^* {\mathfrak{A}_d^*}^{N-2} \mathfrak{P} \mathfrak{B}_d \\
            \vdots & \vdots & \ddots & \vdots \\
            \mathfrak{B}_d^* \mathfrak{P} {\mathfrak{A}_d}^{N-1} \mathfrak{B}_d & \mathfrak{B}_d^* \mathfrak{P} {\mathfrak{A}_d}^{N-2} \mathfrak{B}_d & \cdots & \mathfrak{B}_d^* \mathfrak{P} \mathfrak{B}_d + \mathfrak{F}
        \end{bmatrix} \\
        P = &\begin{bmatrix}
            \mathfrak{B}_d^* \mathfrak{P} {\mathfrak{A}_d} &
            \mathfrak{B}_d^* \mathfrak{P} {\mathfrak{A}_d}^{2}  &
            \hdots &
            \mathfrak{B}_d^* \mathfrak{P} {\mathfrak{A}_d}^{N} 
        \end{bmatrix}^T \\
        T_u (\cdot) = &\begin{bmatrix}
            \langle {\mathfrak{A}_d}^{N} (\cdot), \underline{\phi_u} \rangle
        \end{bmatrix} \\
        S_u = &\begin{bmatrix}
            \langle {\mathfrak{A}_d}^{N-1} \mathfrak{B}_d, \underline{\phi_u} \rangle & 
            \langle {\mathfrak{A}_d}^{N-2} \mathfrak{B}_d, \underline{\phi_u} \rangle &
            \hdots &
            \langle \mathfrak{B}_d, \underline{\phi_u} \rangle
        \end{bmatrix} \\
        U = &\begin{bmatrix}
            u(k+1|k) & u(k+2|k) & \hdots & u(k+N|k)
        \end{bmatrix}^T
    \end{aligned}
\end{equation}