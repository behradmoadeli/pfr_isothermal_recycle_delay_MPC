\section*{Appendix}


\subsection{Adjoint system} \label{app:adjoint}

In this part, the adjoint system operators $\mathfrak{A}^*$ and $\mathfrak{B}^*$ are obtained. Utilizing the relation $\langle \mathfrak{A} \underline{x} + \mathfrak{B} u, \underline{y}\rangle = \langle \underline{x}, \mathfrak{A}^* \underline{y}\rangle + \langle u, \mathfrak{B}^* \underline{y}\rangle$, the adjoint operators $\mathfrak{A}^*$ and $\mathfrak{B}^*$ are obtained as shown in \eqref{eq:adjoint_A} and \eqref{eq:adjoint_B}, respectively.

\begin{equation} \label{eq:adjoint_A}
    \begin{aligned}
        {\mathfrak{A}}^{*} =&
        \begin{bmatrix}
            D \partial_{\zeta \zeta} + v \partial_\zeta +k_r & 0\\
            0 & -\frac{1}{\tau} \partial_\zeta
        \end{bmatrix}\\
        \mathcal{D}(\mathfrak{A}^*) =& \Bigl\{ \underline{y} = [y_1, y_2]^T \in Y:\\
        &\underline{y}(\zeta), \partial_\zeta \underline{y}(\zeta), \partial_{\zeta \zeta} \underline{y}(\zeta) \quad \mathrm{a.c.},\\
        &D \partial_\zeta y_1(1) + v y_1(1) = \frac{1}{\tau} y_2(1), \\
        &R v y_1(0) = \frac{1}{\tau} y_2(0), 
        \partial_\zeta y_1(0) = 0 \Bigr\}
    \end{aligned}
\end{equation}

\begin{equation} \label{eq:adjoint_B}
    \mathfrak{B}^* (\cdot) = \Bigl[ v(1-R) \int_0^1 \delta(\zeta) (\cdot) d\zeta \quad , \quad 0 \Bigr]
\end{equation}

Once the adjoint operators are determined, the eigenfunctions $\{ \underline{\phi_i}(\zeta), \underline{\psi_i}(\zeta) \}$ (for $\mathfrak{A}$ and $\mathfrak{A}^*$, respectively) may be obtained and properly scaled following the calculation of eigenvalues. The set of scaled eigenfunctions will then form a bi-orthonormal basis for the Hilbert space $X$; which will be later used in the controller design. It is important to note that the system is not self adjoint, as the obtained adjoint operator and its domain are not the same as the original operator and its domain.

\subsection{Resolvent Operator Derivation} \label{app:resolvent}

One must obtain the resolvent operator of the system $\mathfrak{R}(s, \mathfrak{A}) = (sI-\mathfrak{A})^{-1}$ prior to constructing the discrete-time representation of the system. One way to obtain it is by utilizing the modal characteristics of the system, resulting in an infinite-sum representation of the operator. While being a common practice in the literature, truncating the infinite-sum representation for numerical implementation may lead to a loss of accuracy. Another way to express the resolvent operator is by treating it as an operator that maps either the initial condition of the system $\underline{x}(\zeta,0)$ or the input $u(t)$, to the Laplace transform of the state of the system $\underline{X}(\zeta, s)$. This approach, although more computationally intensive, results in a closed form expression for the resolvent operator, preserving the infinite-dimensional nature of the system. In \eqref{eq:resolvent}, Laplace transform is applied to the LTI representation of the system for both zero-input response and zero-state response to obtain a general expression for the resolvent operator.

\begin{equation} \label{eq:resolvent}
    \begin{aligned}
        \dot{\underline{x}}(\zeta, t) &= \mathfrak{A} \underline{x}(\zeta, t) + \mathfrak{B} u(t) \xrightarrow{\mathcal{L}}\\
        s \underline{X}(\zeta,s) - \underline{x}(\zeta,0) &= \mathfrak{A} \underline{X}(\zeta,s) + \mathfrak{B} U(s)\\
        &\hspace{-7.5em}\begin{cases}
            \xrightarrow{u = 0} &\underline{X}(\zeta,s) = (sI - \mathfrak{A})^{-1} \underline{x}(\zeta,0) = \mathfrak{R}(s, \mathfrak{A}) \underline{x}(\zeta,0)\\
            \xrightarrow{\underline{x}(0, \zeta)}& \underline{X}(\zeta,s) = (sI - \mathfrak{A})^{-1} \mathfrak{B} U(s) = \mathfrak{R}(s, \mathfrak{A}) \mathfrak{B} U(s)
        \end{cases}
    \end{aligned}
    \end{equation}
    
    The goal is to obtain the solution for $\underline{X}(\zeta, s)$ and compare it with the general expression obtained in \eqref{eq:resolvent} to get the closed form expression for the resolvent operator. First step is to apply Laplace transform to the original system of PDEs in \eqref{eq:operator_A}. The second order derivative term is decomposed to two first order PDEs, constructing a new $3 \times 3$ system of first order ODEs with respect to $\zeta$ after Laplace transformation, as shown in \eqref{eq:laplace_transformed}.
    
    \begin{equation} \label{eq:laplace_transformed}
    \begin{aligned}
        \partial_\zeta \overbrace{\begin{bmatrix}
            X_1(\zeta,s)\\ \partial_\zeta X_1(\zeta,s)\\ X_2(\zeta,s)
        \end{bmatrix}}^{\underline{\tilde{X}}(\zeta,s)} &= \overbrace{\begin{bmatrix}
            0 & 1 & 0\\
            \frac{s-k}{D} & \frac{v}{D} & 0\\
            0 & 0 & s\tau
            \end{bmatrix}}^{P(s)} \, \begin{bmatrix}
                X_1(\zeta,s)\\ \partial_\zeta X_1(\zeta,s)\\ X_2(\zeta,s)
            \end{bmatrix} \\
            &+ \underbrace{\begin{bmatrix}
                0\\ -\frac{x_1(\zeta,0)}{D} + v(1-R) \delta(\zeta) U(s)\\ -\tau x_2(\zeta,0)
            \end{bmatrix}}_{Z(\zeta,s)} \\
            \Rightarrow \partial_\zeta \underline{\tilde{X}}(\zeta,s) &= P(s) \underline{\tilde{X}}(\zeta,s) + Z(\zeta,s)
    \end{aligned}
    \end{equation} 
    
    with solution given by \eqref{eq:ODE_solution}.
    
    \begin{equation} \label{eq:ODE_solution}
        \underline{\tilde{X}}(\zeta,s) = \underbrace{e^{P(s)\zeta}}_{T(\zeta,s)} \underline{\tilde{X}}(0,s) + \int_0^\zeta \underbrace{e^{P(s)(\zeta - \eta)}}_{F(\zeta, \eta)} Z(\eta,s) d\eta
    \end{equation}
    
    Since the boundary conditions are not homogeneous, $\underline{\tilde{X}}(0,s)$ needs to be obtained by solving the system of algebraic equations given in \eqref{eq:BC_AE}; which is the result of applying Danckwerts boundary conditions to the Laplace transformed system of PDEs at $\zeta = 1$.
    
    \begin{equation} \label{eq:BC_AE}
    \begin{aligned}
            &\overbrace{\begin{bmatrix}
                -v & D & Rv\\
                T_{11}(1,s) & T_{12}(1,s) & -T_{33}(1,s)\\
                T_{21}(1,s) & T_{22}(1,s) & 0
            \end{bmatrix}}^{M^{-1}(s)} \underline{\tilde{X}}(0,s) =\\ 
            &\underbrace{\int_0^1 \begin{bmatrix}
                0\\ F_{33}(1, \eta) Z_3(\eta,s) - F_{12}(1, \eta) Z_2(\eta,s)\\ -F_{22}(1, \eta) Z_2(\eta,s)
            \end{bmatrix} d\eta}_{\underline{b}(s)} \\
            \Rightarrow &\underline{\tilde{X}}(0,s) = M(s) \underline{b}(s)
    \end{aligned}
    \end{equation}
    
    Having access to $\underline{\tilde{X}}(0,s)$, the solution for $\underline{X}(\zeta,s)$ can be explicitly derived. The resolvent operator for zero-input and zero-state cases are therefore obtained in a closed form as shown in \eqref{eq:resolvent_x} and \eqref{eq:resolvent_u}, respectively.
    
    \begin{equation} \label{eq:resolvent_x}
    \begin{aligned}
        &U(s) = 0 \Rightarrow \mathfrak{R}(s, \mathfrak{A}) \underline{(\cdot)} = \begin{bmatrix}
            \mathfrak{R}_{11} & \mathfrak{R}_{12}\\
            \mathfrak{R}_{21} & \mathfrak{R}_{22}
        \end{bmatrix} \begin{bmatrix}
            (\cdot)_1\\ (\cdot)_2
        \end{bmatrix} \Rightarrow\\
        &\mathfrak{R}_{11} = \sum_{j=1}^2 \frac{T_{1j}(\zeta)}{D} \int_0^1 \left[ M_{j2} F_{12}(1,\eta) + M_{j3} F_{22}(1,\eta) \right] (\cdot)_1 d\eta\\
        &\hspace{2.5em} -\frac{1}{D} \int_0^{\zeta} F_{12}(\zeta, \eta) (\cdot)_1 d\eta\\
        &\mathfrak{R}_{12} = \sum_{j=1}^2 -\tau T_{1j}(\zeta) \int_0^1 M_{j2} F_{33}(1,\eta) (\cdot)_2 d\eta\\
        &\mathfrak{R}_{21} = \frac{T_{33}(\zeta)}{D} \int_0^1 \left[ M_{32} F_{12}(1,\eta) + M_{33} F_{22}(1,\eta) \right] (\cdot)_1 d\eta\\
        &\mathfrak{R}_{22} = -\tau T_{33}(\zeta) \int_0^1 M_{32} F_{33}(1,\eta) (\cdot)_2 d\eta\\
        &\hspace{2.5em} -\tau \int_0^{\zeta} F_{33}(\zeta, \eta) (\cdot)_2 d\eta
    \end{aligned}
    \end{equation}
    
    \begin{equation} \label{eq:resolvent_u}
    \begin{aligned}
        &\underline{x}(\zeta,0) = 0 \Rightarrow \mathfrak{R}(s, \mathfrak{A}) \mathfrak{B} (\cdot) = \begin{bmatrix}
            \mathfrak{R}_{1} \mathfrak{B}\\
            \mathfrak{R}_{2} \mathfrak{B}
        \end{bmatrix} (\cdot) \Rightarrow\\
        &\mathfrak{R}_{1} \mathfrak{B} = -v(1-R) \bigl[ \sum_{j=1}^{2} T_{1j}(\zeta) (M_{j2} T_{12}(1) + M_{j3} T_{22}(1)) \\
        &\hspace{3em} - T_{12}(\zeta) \bigr] (\cdot)\\
        &\mathfrak{R}_{2} \mathfrak{B} = -v(1-R) \left[ T_{33}(\zeta) (M_{32} T_{12}(1) + M_{33} T_{22}(1)) \right] (\cdot)
    \end{aligned}
    \end{equation}

Since the system generator $\mathfrak{A}$ is not self-adjoint, the resolvent operator for the adjoint system shall also be obtained. This is done in a similar manner as the original system, resulting in a closed-form expression for the adjoint resolvent operator $\mathfrak{R}^*(s, \mathfrak{A}^*)$. To avoid redundancy, the derivation of the resolvent operator for the adjoint system is not included in this manuscript.

% \subsection{Resolvent Operator Derivation} \label{app:resolvent}

% The resolvent is defined as the operator that maps either the initial condition or the input of the system to the Laplace transform of the state, as shown in \eqref{eq:resolvent}.

% \begin{equation} \label{eq:resolvent}
% \begin{aligned}
%     \dot{\underline{x}}(\zeta, t) &= \mathfrak{A} \underline{x}(\zeta, t) + \mathfrak{B} u(t) \xrightarrow{\mathcal{L}}\\
%     s \underline{X}(\zeta,s) - \underline{x}(\zeta,0) &= \mathfrak{A} \underline{X}(\zeta,s) + \mathfrak{B} U(s)\\
%     &\hspace{-7.5em}\begin{cases}
%         u = 0 \Rightarrow \underline{X}(\zeta,s) = (sI - \mathfrak{A})^{-1} \underline{x}(\zeta,0) = \mathfrak{R}(s, \mathfrak{A}) \underline{x}(\zeta,0)\\
%         x = 0 \Rightarrow \underline{X}(\zeta,s) = (sI - \mathfrak{A})^{-1} \mathfrak{B} U(s) = \mathfrak{R}(s, \mathfrak{A}) \mathfrak{B} U(s)
%     \end{cases}
% \end{aligned}
% \end{equation}

% First step to obtain the solution for $\underline{X}(\zeta, s)$ is to apply Laplace transform to the original system of PDEs in \eqref{eq:operator_A} to end up with a new $3 \times 3$ system of ODEs as shown in \eqref{eq:laplace_transformed}.

% \begin{equation} \label{eq:laplace_transformed}
% \begin{aligned}
%     \partial_\zeta \overbrace{\begin{bmatrix}
%         X_1(\zeta,s)\\ \partial_\zeta X_1(\zeta,s)\\ X_2(\zeta,s)
%     \end{bmatrix}}^{\underline{\tilde{X}}(\zeta,s)} &= \overbrace{\begin{bmatrix}
%         0 & 1 & 0\\
%         \frac{s-k}{D} & \frac{v}{D} & 0\\
%         0 & 0 & s\tau
%         \end{bmatrix}}^{P(s)} \, \begin{bmatrix}
%             X_1(\zeta,s)\\ \partial_\zeta X_1(\zeta,s)\\ X_2(\zeta,s)
%         \end{bmatrix} \\
%         &+ \underbrace{\begin{bmatrix}
%             0\\ -\frac{x_1(\zeta,0)}{D} + v(1-R) \delta(\zeta) U(s)\\ -\tau x_2(\zeta,0)
%         \end{bmatrix}}_{Z(\zeta,s)} \\
%         \Rightarrow \partial_\zeta \underline{\tilde{X}}(\zeta,s) &= P(s) \underline{\tilde{X}}(\zeta,s) + Z(\zeta,s)
% \end{aligned}
% \end{equation} 

% The solution for the obtained ODE is given by \eqref{eq:ODE_solution}.

% \begin{equation} \label{eq:ODE_solution}
%     \underline{\tilde{X}}(\zeta,s) = \underbrace{e^{P(s)\zeta}}_{T(\zeta,s)} \underline{\tilde{X}}(0,s) + \int_0^\zeta \underbrace{e^{P(s)(\zeta - \eta)}}_{F(\zeta, \eta)} Z(\eta,s) d\eta
% \end{equation}

% Since the boundary conditions are not homogeneous, $\underline{\tilde{X}}(0,s)$ needs to be obtained by solving the system of algebraic equations given in \eqref{eq:BC_AE}; which is the result of applying Danckwerts boundary conditions to the Laplace transformed system of PDEs at $\zeta = 1$.

% \begin{equation} \label{eq:BC_AE}
% \begin{aligned}
%         &\overbrace{\begin{bmatrix}
%             -v & D & Rv\\
%             T_{11}(1,s) & T_{12}(1,s) & -T_{33}(1,s)\\
%             T_{21}(1,s) & T_{22}(1,s) & 0
%         \end{bmatrix}}^{M^{-1}(s)} \underline{\tilde{X}}(0,s) =\\ 
%         &\underbrace{\int_0^1 \begin{bmatrix}
%             0\\ F_{33}(1, \eta) Z_3(\eta,s) - F_{12}(1, \eta) Z_2(\eta,s)\\ -F_{22}(1, \eta) Z_2(\eta,s)
%         \end{bmatrix} d\eta}_{\underline{b}(s)} \\
%         \Rightarrow &\underline{\tilde{X}}(0,s) = M(s) \underline{b}(s)
% \end{aligned}
% \end{equation}

% Having access $\underline{\tilde{X}}(0,s)$, the solution for $\underline{X}(\zeta,s)$ can be explicitly derived. The resolvent operator for zero-input and zero-state cases are therefore obtained in a closed form as shown in \eqref{eq:resolvent_x} and \eqref{eq:resolvent_u}, respectively.

% \begin{equation} \label{eq:resolvent_x}
% \begin{aligned}
%     &U(s) = 0 \Rightarrow \mathfrak{R}(s, \mathfrak{A}) \underline{(\cdot)} = \begin{bmatrix}
%         \mathfrak{R}_{11} & \mathfrak{R}_{12}\\
%         \mathfrak{R}_{21} & \mathfrak{R}_{22}
%     \end{bmatrix} \begin{bmatrix}
%         (\cdot)_1\\ (\cdot)_2
%     \end{bmatrix} \Rightarrow\\
%     &\mathfrak{R}_{11} = \sum_{j=1}^2 \frac{T_{1j}(\zeta)}{D} \int_0^1 \left[ M_{j2} F_{12}(1,\eta) + M_{j3} F_{22}(1,\eta) \right] (\cdot)_1 d\eta\\
%     &\hspace{2.5em} -\frac{1}{D} \int_0^{\zeta} F_{12}(\zeta, \eta) (\cdot)_1 d\eta\\
%     &\mathfrak{R}_{12} = \sum_{j=1}^2 -\tau T_{1j}(\zeta) \int_0^1 M_{j2} F_{33}(1,\eta) (\cdot)_2 d\eta\\
%     &\mathfrak{R}_{21} = \frac{T_{33}(\zeta)}{D} \int_0^1 \left[ M_{32} F_{12}(1,\eta) + M_{33} F_{22}(1,\eta) \right] (\cdot)_1 d\eta\\
%     &\mathfrak{R}_{22} = -\tau T_{33}(\zeta) \int_0^1 M_{32} F_{33}(1,\eta) (\cdot)_2 d\eta\\
%     &\hspace{2.5em} -\tau \int_0^{\zeta} F_{33}(\zeta, \eta) (\cdot)_2 d\eta
% \end{aligned}
% \end{equation}

% \begin{equation} \label{eq:resolvent_u}
% \begin{aligned}
%     &\underline{x}(\zeta,0) = 0 \Rightarrow \mathfrak{R}(s, \mathfrak{A}) \mathfrak{B} (\cdot) = \begin{bmatrix}
%         \mathfrak{R}_{1} \mathfrak{B}\\
%         \mathfrak{R}_{2} \mathfrak{B}
%     \end{bmatrix} (\cdot) \Rightarrow\\
%     &\mathfrak{R}_{1} \mathfrak{B} = -v(1-R) \bigl[ \sum_{j=1}^{2} T_{1j}(\zeta) (M_{j2} T_{12}(1) + M_{j3} T_{22}(1)) \\
%     &\hspace{3em} - T_{12}(\zeta) \bigr] (\cdot)\\
%     &\mathfrak{R}_{2} \mathfrak{B} = -v(1-R) \left[ T_{33}(\zeta) (M_{32} T_{12}(1) + M_{33} T_{22}(1)) \right] (\cdot)
% \end{aligned}
% \end{equation}

% Same procedure can be applied to obtain the closed form expression for resolvent operator of the adjoint system. To avoid redundancy, the details are not provided here.