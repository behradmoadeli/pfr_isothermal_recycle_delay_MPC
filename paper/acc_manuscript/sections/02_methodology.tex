\section{Methodology}

\subsection{Model representation}

The chemical process depicted in Fig.~\ref{fig:reactor_scheme} illustrates a first-order irreversible chemical reaction within an axial dispersion tubular reactor \cite{levenspiel1998chemical}. The reactor features a recycle mechanism, allowing a portion of the product stream to re-enter the reactor, ensuring the consumption of any unreacted substrate. The reactor's dynamics can be described by a second-order parabolic PDE, a common class of equations used to characterize diffusion-convection-reaction systems \cite{jensen1982bifurcation}. The resulting PDE that describes the reactor model is given by (\ref{eq:PDE_original_model}), subject to the boundary conditions in (\ref{eq:BC}), obtained by utilizing first-principle modeling through relevant mass balance relations on an infinitesimally small section of the reactor.

\begin{figure}[!htbp] 
    \centering
    \begin{tikzpicture}
        \node (pfr) [cylinder, draw, minimum height=3cm, minimum width=1cm, shape aspect=1, shape border rotate=180, cylinder uses custom fill, cylinder end fill=gray!45, cylinder body fill=gray!15] {$\underline{A} \rightarrow \underline{B}$};
        \node (pfr_inlet) [circle, left of=pfr, xshift=-0.5cm, fill=black, draw, inner sep=0pt, minimum size=0.25cm, scale=0.5] {};
        \node (pfr_outlet) [circle, at={(pfr.east)}, shift={(-0.25cm,0)}, fill=black, draw, inner sep=0pt, minimum size=0.25cm, scale=0.5] {};
        \node (recycle_right) [circle, right of=pfr_outlet, fill=black, draw, inner sep=0pt, minimum size=0.25cm, scale=0.5] {};
        \node (recycle_left) [circle, left of=pfr_inlet, fill=black, draw, inner sep=0pt, minimum size=0.25cm, scale=0.5] {};
        
        \draw[dotted, thick] ([yshift=0.5cm]pfr_inlet.center) -- node[at end, below, yshift=0.1cm] {$\zeta = 0$} ([yshift=-0.65cm]pfr_inlet.center);
        \draw[dotted, thick] ([yshift=0.5cm]pfr_outlet.center) -- node[at end, below, yshift=0.1cm] {$\zeta = 1$} ([yshift=-0.65cm]pfr_outlet.center);
        
        \node[below of=recycle_left, node distance=1.3cm, anchor=north west, xshift=-0.2cm] {$R \, c(1, t-\tau)$};
        \node[above of=pfr_inlet, node distance=0.75cm,] {$c(0, t)$};
        \node[above of=pfr_outlet, node distance=0.75cm,] {$c(1, t)$};
        
        \draw [arrow_2] (pfr_outlet) -- node[near end, above] {$y(t)$} ++(2,0);
        \draw [arrow_2] (pfr_inlet) ++(-2,0) coordinate(start) -- node[near start, above] {$u(t)$} (pfr_inlet);
        \draw [arrow_2] (recycle_right) -- ++(0,-1.25) -| (recycle_left);
        
    \end{tikzpicture}
    \caption{Axial tubular reactor with recycle stream.}
    \label{fig:reactor_scheme}
    % \pdfbookmark[2]{Figure: Reactor Scheme}{fig:reactor_scheme}
\end{figure}

\begin{equation} \label{eq:PDE_original_model}
    \dot{c}(\zeta, t) = D \partial_{\zeta \zeta} c(\zeta, t) - v \partial_\zeta c(\zeta, t) + k_r c(\zeta, t)
\end{equation}

\begin{align} \label{eq:BC}
    \begin{cases}
        &D \partial_\zeta c(0, t) - v c(0, t) = -v \left[ R c(1, t-\tau) + (1-R) u(t) \right] \\
        &\partial_\zeta c(1, t) = 0 \\
        &y(t) = c(1, t)
    \end{cases}
\end{align}

Here, $c(\zeta, t)$ is the concentration of the product along the reactor, representing the state of the system. The physical parameters $D$, $v$, $k_r$, $R$, and $\tau$ represent the diffusion coefficient, flow velocity along the reactor, reaction constant, recycle ratio, and residence time of the recycle flow, respectively. The coordinate system in space and time is represented by $\zeta$ and $t$, where $\zeta \in [0, 1]$ and $t \in [0, \infty)$.

In an attempt to make the model more realistic for common axial dispersion tubular reactors in chemical industry, Dankwerts boundary conditions are chosen as they are known to be suitable for this purpose by accounting for deviations from perfect mixing and piston flow, assuming negligible transport lags in connecting lines \cite{danckwerts1993continuous}. The delayed state resulting from the recycled portion of the flow, occurring $\tau$ seconds back in time, is applied at the inlet boundary condition, as shown in (\ref{eq:BC}).

In the case where the problem involves similar forms of PDEs, an effective general practice to address delays in systems is to reformulate the problem such that the notion of delay is replaced with an alternative transport PDE. Therefore, a new state variable $\underline{x}(\zeta, t)$ is defined as a vector of functions $\equiv [x_1(\zeta, t), x_2(\zeta, t)]^T$, where $x_1(\zeta, t)$ represents the concentration within the reactor—analogous to $c(\zeta,t)$—and $x_2(\zeta, t)$ is introduced as a new state variable to account for the concentration along the recycle stream. The delay is thus modeled as a pure transport process, wherein the first state $x_1(\zeta, t)$ is transported from the reactor outlet to the inlet, experiencing a delay of $\tau$ time units while in the recycle stream. This makes all state variables expressed explicitly at a specific time instance $t$, resulting in the standard state-space form for a given infinite-dimensional linear time-invariant (LTI) system as $\dot{\underline{x}} = \mathfrak{A} \underline{x} + \mathfrak{B} u$. Here, $\mathfrak{A}$ is a linear operator $\mathcal{L}(X)$ acting on a Hilbert space $X: L^2[0,1] \times L^2[0,1]$ and $\underline{x}(\zeta,t)$, as defined previously, is the vector of functions describing the states of the system. The operator $\mathfrak{A}$ and its domain are defined in detail as shown in (\ref{eq:operator_A}). Also, $\mathfrak{B}$ is a linear operator that maps the scalar input from input-space onto the state space, as defined in (\ref{eq:operator_B}).

\begin{equation} \label{eq:operator_A}
    \begin{aligned}
        \mathfrak{A} \equiv&
        \begin{bmatrix}
            D \partial_{\zeta \zeta} - v \partial_\zeta + k_r & 0 \\
            0 & \frac{1}{\tau} \partial_\zeta
        \end{bmatrix}\\
        D(\mathfrak{A}) =& \Bigl\{ \underline{x}(\zeta) = [x_1(\zeta), x_2(\zeta)]^T \in X:\\
        &\underline{x}(\zeta), \partial_\zeta \underline{x}(\zeta), \partial_{\zeta \zeta} \underline{x}(\zeta) \quad \mathrm{a.c.},\\
        &D \partial_\zeta x_1(0) - v x_1(0) = -v R x_2(0),\\
        &\partial_\zeta x_1(1) = 0,
        x_1(1) = x_2(1) \Bigr\}
    \end{aligned}
\end{equation}

\begin{equation} \label{eq:operator_B}
    \begin{aligned}
        \mathfrak{B} &\equiv
        \begin{bmatrix}
            \delta(\zeta) \\
            0
        \end{bmatrix} v (1-R) \\
        D(\mathfrak{B}) &= \Bigl\{ u \in \mathbb{R} \Bigr\}
    \end{aligned}
\end{equation}

where $\delta(\zeta)$ is dirac delta function. This will enable the derivation of the system's spectrum using the eigenvalue problem. The characteristics equation of the system is obtained by solving the equation $det(\mathfrak{A}-\lambda_i~I)~=~0$ for $\lambda_i$, where $\lambda_i \in \mathbb{C}$ is the $i^{\text{th}}$ eigenvalue of the system. Attempts to analytically solve this equation has failed; therefore, it is solved numerically using the parameters in Table~\ref{tab:pars}. The resulting eigenvalue distribution is depicted in Figure~\ref{fig:eigval_dist} in the complex plane.

\begin{figure}[!htbp]
    \centering
    \includesvg[inkscapelatex=false, width=0.35\textwidth, keepaspectratio]{Figures/eig_val_dist_R_0.3.svg}
    \caption{Eigenvalues of operator $\mathfrak{A}$.}
    \label{fig:eigval_dist}
\end{figure}


\begin{table}[ht]
    \centering
    \caption{Physical Parameters for the System}
    \label{tab:pars}
    \begin{tabular}{|c|c|c|c|}
    \hline
    \textbf{Parameter}        & \textbf{Symbol} & \textbf{Value}     & \textbf{Unit}    \\ \hline
    Diffusivity               & $D$             & $2\times10^{-5}$   & ${m^2}/{s}$      \\ \hline
    Velocity                  & $v$             & $0.01$   & ${m}/{s}$        \\ \hline
    Reaction Constant         & $k_r$           & $1.5$              & $s^{-1}$         \\ \hline
    Recycle Residence Time    & $\tau$          & $80$               & $s$              \\ \hline
    Recycle Ratio             & $R$             & $0.3$              & $-$              \\ \hline
    \end{tabular}
\end{table}

\subsection{Adjoint system}

Next step is to obtain the adjoint system operators $\mathfrak{A}^*$ and $\mathfrak{B}^*$. Utilizing the relation $\langle \mathfrak{A} \underline{x} + \mathfrak{B} u, \underline{y}\rangle = \langle \underline{x}, \mathfrak{A}^* \underline{y}\rangle + \langle u, \mathfrak{B}^* \underline{y}\rangle$, the adjoint operators $\mathfrak{A}^*$ and $\mathfrak{B}^*$ are obtained as shown in (\ref{eq:adjoint_A}) and (\ref{eq:adjoint_B}), respectively.


\begin{equation} \label{eq:adjoint_A}
    \begin{aligned}
        {\mathfrak{A}}^{*} =&
        \begin{bmatrix}
            D \partial_{\zeta \zeta} + v \partial_\zeta +k_r & 0\\
            0 & -\frac{1}{\tau} \partial_\zeta
        \end{bmatrix}\\
        D(\mathfrak{A}^*) =& \Bigl\{ \underline{y} = [y_1, y_2]^T \in Y:\\
        &\underline{y}(\zeta), \partial_\zeta \underline{y}(\zeta), \partial_{\zeta \zeta} \underline{y}(\zeta) \quad \mathrm{a.c.},\\
        &D \partial_\zeta y_1(1) + v y_1(1) = \frac{1}{\tau} y_2(1), \\
        &R v y_1(0) = \frac{1}{\tau} y_2(0), 
        \partial_\zeta y_1(0) = 0 \Bigr\}
    \end{aligned}
\end{equation}

\begin{equation} \label{eq:adjoint_B}
    \mathfrak{B}^* (\cdot) = \Bigl[ v(1-R) \int_0^1 \delta(\zeta) (\cdot) d\zeta \quad , \quad 0 \Bigr]
\end{equation}

Once the adjoint operators are determined, the eigenfunctions $\{ \underline{\phi_i}(\zeta), \underline{\psi_i}(\zeta) \}$ (for $\mathfrak{A}$ and $\mathfrak{A}^*$, respectively) may be obtained and properly scaled following the calculation of eigenvalues. The set of scaled eigenfunctions will then form a bi-orthonormal basis for the Hilbert space $X$; which will be later used in the controller design. It is important to note that the system is not self adjoint, as the obtained adjoint operator and its domain are not the same as the original operator and its domain.

\subsection{Resolvent operator}

One must obtain the resolvent operator of the system prior to constructing the discrete-time representation of the system. One way to represent the resolvent operator is by utilizing the modal characteristics of the system, resulting in an infinite-sum representation of the operator. While being a common practice in the literature, truncating the infinite-sum representation for numerical implementation may lead to a loss of accuracy. Another way to express the resolvent operator is by treating it as an operator that maps either the initial condition of the system $\underline{x}(\zeta,0)$ or the input $u(t)$, to the laplace transform of the state of the system $\underline{X}(\zeta, s)$. This approach, although more computationally intensive, results in a closed form expression for the resolvent operator, preserving the infinite-dimensional nature of the system. The resolvent operator $\mathfrak{R}(s, \mathfrak{A}) = (sI-\mathfrak{A})^{-1}$ is therefore obtained by applying laplace transform to the LTI representation of the system for both for zero-input response and zero-state response, and solving for $\underline{X}(\zeta, s)$.
% The procedure as well as the resulting closed form expression for the resolvent operator is shown in Appendix~\ref{app:resolvent}.

Since the system generator $\mathfrak{A}$ is not self-adjoint, the resolvent operator for the adjoint system shall also be obtained. This is done in a similar manner as the original system, resulting in a closed-form expression for the adjoint resolvent operator $\mathfrak{R}^*(s, \mathfrak{A}^*)$.
% as shown in Appendix~\ref{app:resolvent}.

\subsection{Caley-Tustin time discretization}

Having access to the resolvent operators of the original and the adjoint system, the Caley-Tustin time-discretization may be utilized to map the continuous-time setting to the discrete-time setting without losing crucial dynamical properties of the system, such as stability and controllability. This Crank-Nicolson type of discretization is also known as the lowest order symplectic integrator in Gauss quadrature-based Runge-Kutta methods \cite{hairer2006geometric}. Considering $\Delta t$ as the sampling time, and assuming a piecewise constant input within time intervals (a.k.a. zero-order hold), the discrete-time representation $\underline{x}(\zeta, k) = \mathfrak{A}_d \underline{x}(\zeta, k-1) + \mathfrak{B}_d u(k)$ is obtained, with discrete-time operators $\mathfrak{A}_d$ and $\mathfrak{B}_d$ defined in (\ref{eq:discrete_AB}), where $\alpha = 2/{\Delta t}$.

\begin{equation} \label{eq:discrete_AB}
    \begin{bmatrix}
        \mathfrak{A}_d & \mathfrak{B}_d
    \end{bmatrix} = 
    \begin{bmatrix}
        -I + 2\alpha \mathfrak{R}(\alpha, \mathfrak{A}) & \sqrt{2\alpha} \mathfrak{R}(\alpha, \mathfrak{A}) \mathfrak{B}
    \end{bmatrix}
\end{equation}

As required for systems with nonself-adjoint generators, the adjoint discrete-time operators $\mathfrak{A}_d^*$ and $\mathfrak{B}_d^*$ are also obtained in a similar manner, as shown in (\ref{eq:discrete_AB_star}).

\begin{equation} \label{eq:discrete_AB_star}
    \begin{bmatrix}
        \mathfrak{A}_d^* & \mathfrak{B}_d^*
    \end{bmatrix} = 
    \begin{bmatrix}
        -I + 2\alpha \mathfrak{R}^*(\alpha, \mathfrak{A}^*) & \sqrt{2\alpha} \mathfrak{B}^* \mathfrak{R}^*(\alpha, \mathfrak{A}^*)
    \end{bmatrix}
\end{equation}

\subsection{Model predictive control design}

The proposed MPC, as shown in Fig.~\ref{fig:block_diagram}, is developed in this section with the goal of stabilizing the given unstable infinite-dimensional system within an optimal framework while satisfying input constraints. An infinite-time open-loop objective function sets the foundation of the controller design in the discrete-time setting at each sampling instant $k$, which consists of a weighted sum of state deviations and actuation costs for all future time instances, subject to the system dynamics and input constraints, as shown in (\ref{eq:MPC_inf_time}).

\begin{figure}[!htbp]
    \centering
    \begin{tikzpicture}[node distance=2cm]
        \node (plant) [block] {Plant};
        \node (MPC) [block, below of=plant] {MPC};
        \draw [arrow] (plant.south) -- node[midway, right] {$\underline{x}(\zeta,k)$} (MPC.north);
        \draw [arrow] (MPC.west) -- ++(-1,0) |- node[near end, above] {$u(k)$} (plant.west);
        \draw [arrow] (plant.east) -- node[midway, above] {$y(k)$} ++(1,0);
    \end{tikzpicture}
    \caption{Proposed full-state feedback model predictive control system.}
    \label{fig:block_diagram}
\end{figure}

\begin{equation} \label{eq:MPC_inf_time}
    \begin{aligned}
        \min_{U} \quad \sum_{l=0}^{\infty} &\langle \underline{x}(\zeta, k+l | k), \mathfrak{Q} \underline{x}(\zeta, k+l | k) \rangle \\
        + &\langle u(k+l+1 | k), \mathfrak{F} u(k+l+1|l) \rangle \\
        \, \\
        \text{s.t.} \quad &\underline{x}(\zeta, k+l | k) = \mathfrak{A}_d \underline{x}(\zeta, k+l-1 | k) + \mathfrak{B}_d u(k+l | k) \\
        &u^{min} \leq u(k+l | k) \leq u^{max}
    \end{aligned}
\end{equation}

where $\mathfrak{Q}$ and $\mathfrak{F}$ are positive definite operators of appropriate dimensions, responsible for penalizing state deviations and actuation costs, respectively. The notation $(k+l|k)$ indicates the future time states or input instance $k+l$ obtained at time $k$. The infinite-time optimization problem may be reduced to a finite-time setup by assigning zero-input beyond a certain control horizon $N$, result in the optimization problem in (\ref{eq:MPC_finite_time}).

\begin{equation} \label{eq:MPC_finite_time}
    \begin{aligned}
        \min_{U} \quad \sum_{l=0}^{N-1} &\langle \underline{x}(\zeta, k+l | k), \mathfrak{Q} \underline{x}(\zeta, k+l | k) \rangle \\
        + &\langle u(k+l+1 | k), \mathfrak{F} u(k+l+1|l) \rangle \\
        + &\langle \underline{x}(\zeta, k+N | k), \mathfrak{P} \underline{x}(\zeta, k+N | k) \rangle \\
        \, \\
        \text{s.t.} \quad &\underline{x}(\zeta, k+l | k) = \mathfrak{A}_d \underline{x}(\zeta, k+l-1 | k) + \mathfrak{B}_d u(k+l | k) \\
        &u^{min} \leq u(k+l | k) \leq u^{max} \\
        & \langle \underline{x}(\zeta, k+N | k), \underline{\phi_u}(\zeta) \rangle = 0
    \end{aligned}
\end{equation}

In (\ref{eq:MPC_finite_time}), $\mathfrak{P}$ is the terminal cost operator, which can be proven to be positive definite only if the terminal state $\underline{x}(\zeta, k+N | k)$ is in a stable subspace. Therefore, an equality constraint is introduced to guarantee that the resulting quadratic optimization problem is convex. The terminal constraint is enforced by setting the projection of the terminal state onto the unstable subspace of the system to zero \cite{xu2017linear, khatibi2021model}. Here, $\underline{\phi_u}(\zeta)$ is the set of unstable eigenfunctions of the system, for all eigenvalues where $\operatorname{Re}(\lambda_u) \geq 0$.

One may further process the optimization problem in (\ref{eq:MPC_finite_time}) to obtain a standard format for quadratic programming (QP) solvers by substituting the future states in terms of the current state and the sequence of future inputs using system dynamics expression. The resulting QP problem is given in (\ref{eq:MPC_QP}). The optimal input sequence $U$ is then obtained by solving the QP problem at each sampling instant $k$. To implement a receding horizon control strategy, only the first input of the optimal sequence $u(k|k)$ is applied to the system, and the optimization problem is solved again at the next sampling instant $k+1$.

\begin{equation} \label{eq:MPC_QP}
    \begin{aligned}
        \min_{U} J = &U^T \langle I,H \rangle U + 2U^T \langle I, P \underline{x}(\zeta, k|k) \rangle \\
        \text{s.t.} \quad &U^{min} \leq U \leq U^{max} \\
        &T_u \underline{x}(\zeta, k|k) + S_u U = 0
        \, \\
        \text{with:} &\\
        &\quad H = \begin{bmatrix}
            \mathfrak{F} & 0 & \cdots & 0 \\
            0 & \mathfrak{F} & \cdots & 0 \\
            \vdots & \vdots & \ddots & \vdots \\
            0 & 0 & \cdots & \mathfrak{P}
        \end{bmatrix} \\
    \end{aligned}
\end{equation}