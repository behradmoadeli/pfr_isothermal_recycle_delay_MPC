\section{Introduction}

Many chemical and petrochemical processes involve states that are distributed in space and time. These systems, a.k.a. distributed parameter systems (DPS), are often modeled using partial differential equations (PDEs) to account for distributed states dynamics. Due to the infinite-dimensional nature of DPSs, the control and estimation of these systems becomes inevitably more challenging compared to the well-developed control theories for finite-dimensional systems \cite{ray1981advanced}, making this field an active and fertile direction of research. Two primary methods have been proposed to approach the control of DPSs in the literature. The first one is \textit{Early Lumping}, i.e. reducing the infinite-dimensional system to a finite-dimensional one by spatial discretization in the early stages of system modeling \cite{davison1976robust}. Approximation of the system dynamics at this stage leads to standard control strategies; however, the accuracy of the model is compromised due to potential mismatch between the original system and the reduced-order model \cite{moghadam2012infinite}. On the contrary, the second method, \textit{Late Lumping}, aims to preserve the infinite-dimensional nature of the system. Approximation may be made only in the final numerical implementation of the controller, resulting in a more accurate yet more complex control strategy.

Numerous studies have utilized a late lumping approach for the purpose of controlling infinite dimensional systems within the field of chemical engineering, focusing on the control of either convection-reaction systems governed by first order hyperbolic PDEs, or diffusion-convection-reaction systems governed by second order parabolic PDEs. 
In \cite{christofides1996feedback}, robust control of first order hyperbolic PDEs was addressed, where a plug flow reactor system is stabilized under a distributed input. 
Boundary feedback stabilization using backstepping method is proposed in \cite{krstic2008backstepping} for a similar system of first order hyperbolic PDEs. 
In \cite{xu2016state}, state feedback regulator design is proposed for a countercurrent heat exchanger system, which is another example of a chemical engineering DPS governed by first order hyperbolic PDEs other than tubular reaction systems.
Introducing the effect of dispersion as a prominent aspect of axial dispersion tubular reactors, the robust control of a diffusion-convection-reaction systems governed by second order parabolic PDEs is studied in \cite{christofides1998robust}. 
A late-lumping based approach is considered in \cite{dubljevic2006predictive2} to design a low-dimensional predictive controller for a diffusion-convection-reaction system, where the dominant modes of the system are captured by modal decomposition. 
Similar approach has been utilized to design an observer-based model predictive controller (MPC) in \cite{khatibi2021model} for an axial dispersion tubular reactor, considering the effects of recycle stream as a well-known feature of industrial chemical reactors.

Delay systems are another class of infinite-dimensional systems that have been studied in the literature \cite{curtainbook}. 
Commonly represented in the form of delay differential equations (DDEs), delay can also be modeled as a transport PDE, showing to be advantageous in more complex scenarios \cite{krstic2009book}. In the field of control theory for chemical engineering DPSs, input/output delay has been addressed in several works as both output measurement delays and input actuation delays are common in industrial processes. 
In general, such delays can be addressed by considering a transportation lag block at either the input or output of the system, resulting in a cascade PDE system \cite{Hiratsuka1969IEEE, mohammadi2012lq, Guilherme2019ACC}. In contrast to input/output delays, state delay is less addressed in the relevant literature, probably since not many applications in this field can be described by state delays. In one of the few attempts, a delayed-state distributed parameter system is addressed in \cite{ozorio2019heat}, where a full-state and output feedback regulator is designed for a system of heat exchangers. The state delay in this works comes from the time it takes for a stream to leave one pass of the heat exchanger and enter the next pass. Similarly in \cite{qi2021output}, a tubular reactor system is considered, where the state delay is introduced as a result of the recycle delay in the system, without considering the diffusion term along the reactor. Even in \cite{khatibi2021model} where the recycle stream is considered for a distributed diffusion-convection-reaction system, the recycle is assumed to be instantaneous; a simplifying assumption that leaves a gap in the literature regarding diffusion-convection-reaction systems with a recycle stream imposing state delay.

In this work, an axial dispersion tubular reactor equipped with recycle is addressed as a diffusion-convection-reaction DPS. First, the reactor is modeled by a second order parabolic PDE, where the recycle stream poses a state delay, resulting in a first order hyperbolic transport PDE coupled with the original PDE. Late lumping approach is utilized by obtaining the resolvent of the infinite-dimensional system in a closed operator form, with no need to perform spatial discretization. Then, to enable the implementation of MPC as a digital controller, discrete-time representation of the system is obtained using Caley-Tustin time discretization technique; i.e. a Crank-Nicolson type of discretization that preserves the conservative characteristics of the continuous system, mitigating the need for model reduction \cite{havu2007cayley, xu2017linear}. Finally via numerical simulations, the proposed controller is shown to stabilize an unstable system within an optimal framework, given input constraints.